\documentclass[12pt]{article}
\usepackage{geometry}                % See geometry.pdf to learn the layout options. There are lots.
\geometry{letterpaper}                   % ... or a4paper or a5paper or ... 
\usepackage{graphicx}
\usepackage{amssymb}
\usepackage{amsthm}
\usepackage{epstopdf}
\usepackage{getfiledate}
\usepackage[utf8]{inputenc}
\usepackage[usenames,dvipsnames]{color}
\usepackage[table]{xcolor}
\usepackage{hyperref}
\DeclareGraphicsRule{.tif}{png}{.png}{`convert #1 `dirname #1`/`basename #1 .tif`.png}

\theoremstyle{definition}
\newtheorem{example}{Example}

\newenvironment{explanation}{%
   \setlength{\parindent}{0pt}
   \itshape
   \color{blue}
}{}

\newcommand{\projectname}{Dahoam}
\newcommand{\productname}{Dahoam}
\newcommand{\projectleader}{I. Gattringer}
\newcommand{\responsible}{B. Schröder}
\newcommand{\lastChanged}{\getfiledate[putprefix=false, notime, datecolor=black]{proposal.tex}}
\newcommand{\documentstatus}{In process}
%\newcommand{\documentstatus}{Submitted}
%\newcommand{\documentstatus}{Released}
\newcommand{\version}{V1.1}





\begin{document}
\begin{titlepage}
\begin{flushright}
\includegraphics[scale=.5]{htlleondinglogo.png}\\
\begin{figure}[h]
\includegraphics[scale=.17]{dahoamLogo.png}
\centering
\end{figure}
\end{flushright}

\vspace{1em}

\begin{center}
{\Huge Project Proposal} \\[3em]
{\LARGE \productname} \\[3em]
\end{center}

\begin{flushleft}
\begin{tabular}{|l|l|}
\hline
Project Name & \projectname & \hline
Project Leader & \projectleader & \hline
Responsible  & \responsible & \hline 
Processing State & \documentstatus & \hline
Version & \version & \hline
Last Changed & \lastChanged & \hline
\end{tabular}
\end{flushleft}

\vspace{0.3cm}

\section*{Further Product Information}
\begin{flushleft}
\begin{tabular}{|l|l|}
\hline
\cellcolor[gray]{0.5}\textcolor{white}{Participating} & {Ivonne Gattringer, Alexander Kapsammer, Manuel Kommenda, Marcel Pölzl} & \hline
\cellcolor[gray]{0.5}\textcolor{white}{Creation} & {Initial Intern} & \hline
\end{tabular}
\end{flushleft}
\end{titlepage}
\tableofcontents
\pagebreak

\section{Introduction} 
\begin{explanation} \color{black} \normalfont
Since Smart Homes are getting more popular in the last few years, a big new market has appeared. Today’s situation is that devices which are managing all the different gadgets and other appliances in houses are needed. In most smart homes there is a small tablet which can be mounted on a wall in the entrance room. Which means an extra space is wasted. \textit{Dahoam} should prevent the waste of extra space and combine a commercial mirror with a smart home hub, despite that a mirror is way more flexible in where to install it.
\end{explanation}
\pagebreak

\section{Initial Situation} 
\begin{explanation} \color{black} \normalfont
An initial prototype of Dahoam was built between 2018 and 2019. \\
 
The mirror is capable of
\begin{itemize}
	\item detecting people via a built-in face recognition
	\item understanding spoken commands
	\item displaying weather, email, calendar
	\item turning on/off lights
\end{itemize}


as well as a “Dahoam-Connect-App” to manage profiles of users. 
Each functionality runs in its own microservice. These services run in their own Docker-container and are handled through a master service. All services are started within a single build.\\ \\
Current problems/bugs are
\begin{itemize}
    \item No aborting when building with errors
    \item Building only with sudo
    \item Face/Speak stops during runtime
    \item When selecting weather intent in the Dahoam-Connect-App, all other panels stop working
    \item CRUD management of users not existing
    \item Complex procedures to setup new Raspberry Pi
    \item Aborting user registration not possible
    \item MQTT-Topics are static
    \item Wheater location is static
    \item QR-Code scanning while in light environment
    \item Logout on mirror client not possible
    \item Displayed mails can't be opened
    \item Emails only via POP3
\end{itemize}

\end{explanation}

\pagebreak

\section{General Conditions and Constraints}

Our know-how:
\begin{itemize}
	\item Java
	\item Angular
	\item C\#
	\item SQL
	\item Typescript
	\item C/C++
	\item MQTT
	\item Unity
\end{itemize}
Constraints:
\begin{itemize}
	\item Hardware prespecified (limitation in performance)
	\item Speech and face recognition run locally (high performance impact)
\end{itemize}

\pagebreak

\section{Project Objectives and System Concepts}

\begin{itemize}
    \item Setup and Build
    \begin{itemize}
        \item \textbf{Enable full dockerized runtime environment} \\
         To simplify the setup of a new Raspberry Pi, every microservice will run in its own docker-container. This will also ease error handling
        \item \textbf{Development environment} \\
        Each microservice will have its own logfile and throw specific exceptions, which will simplify searching for errors.
        All services can be tested individually via Node-RED.
        Building will be possible without "sudo"
    \end{itemize}
    \item Documentation
    \item Dahoam-Administration
    \begin{itemize}
        \item \textbf{Dahoam-Connect-App and web-frontend for browser-clients} \\
        Find an alternative for QR-Code on user registration. \\
        Add the possibility to login via face recognition.
        \item \textbf{User administration} \\
        An admin role will be added which can manage all users via CRUD functionalities. Further there will be a simplified usage of calendar, email and weather location configuration. Encrypting user passwords will add security in saving them in the database.
        \item \textbf{Smart home configuration} \\
        Change smart home configuration from static to dynamic values and add features for IOT integration.
    \end{itemize}
    \item Demo mode 
    \begin{itemize}  \\
    \item \textbf{24/7 runable}\\
        Add kiosk startup system to automatically start the mirror without any additional IO-hardware. The dockerized services can be set to sleep mode to enhance runtime of the mirror and reduce heat problems. An extra movement sensor will detect activities in front of the mirror and wake needed services up again.
    \item \textbf{Easter egg}\\
        Add "easter egg" feature for presentations.
    \item \textbf{Hardware integration}\\
        Integrate speakers into the mirror and add cooling to the Raspberry Pi.\\
    \end{itemize}
    \item Dahoam flows
    \begin{itemize} \\
        \item \textbf{Improve session management}\\
        Only logged-in users can show emails and calendar. Each user must have an unique username and the possibility to log themselves out.
        \item \textbf{Face recognition}\\
        Solve the recognition problem when several registered users stand in front of the mirror.
        \item \textbf{Registration flow}\\
        While registration there is a live camera feedback to help the user placing his/her face in the right position. When the registration is aborted there has to be the possibility to register again after that.
        \item \textbf{Improve speech recognition quality}\\
        Find an alternative speech recognition to "sphinx".
        \item \textbf{Optional}\\
        Make the user dashboard customizable.
    \end{itemize}
\end{itemize}
\pagebreak
Used technology:
\begin{itemize}
    \item Java
    \item Docker
    \item Ionic
    \item Kotlin
    \item Angular
    \item Typescript
    \item Python
    \item Git
    \item MQTT
    \item Nginx
    \item HTML
    \item openHAB
    \item Node RED
    \item phpMyAdmin
\end{itemize}


\pagebreak

\section{Opportunities and Risks}
Opportunities:
Dahoam is one of the leading projects of HTL Leonding. The mirror is exposed on various events and is meant to raise the reputation of the school.
It combines both of the main focuses of HTL Leonding, IOT and machine learning.
\newline
Risks:
\begin{itemize}
    \item Due to space problems, the mirror has limited cooling available for the Raspberry Pi, so a full usage of the resources is not possible for a longer time.
    \item Compatibility issues when switching from Raspberry Pi 3 to Raspberry Pi 4.
\end{itemize}

\pagebreak
\section{Planning}
Start: 6. September 2019
\newline
End: 11. November 2019

\pagebreak
\section{Milestones}

\begin{tabular}{|l|l|}
\hline
\cellcolor[gray]{0.5}\textcolor{white}{Title} & \cellcolor[gray]{0.5}\textcolor{white}{Date} &
{Enable full dockerized runtime environment} & {15. Dec 2019} & \hline
{Development environment} & {} & \hline
{ - Improve logging} & {TADEOT Jan 2020} & \hline
{ - Isolated testing of services with Node-RED} & {TADEOT Jan 2020} & \hline
{Dahoam-Connect-App} & {} & \hline
{ - Alternative to QR-code} & {TADEOT Jan 2020} & \hline
{ - Admin account} & {TADEOT Jan 2020} & \hline
{User administration} & {} & \hline
{ - Full CRUD functionalities} & {TADEOT Jan 2020} & \hline
{ - Login procedure} & {TADEOT Jan 2020} & \hline
{24/7 Demo mode} & {} & \hline
{ - Kiosk for autostart} & {TADEOT Jan 2020} & \hline
{ - "Easter egg" feature} & {TADEOT Jan 2020} & \hline
{ - Sleep mode} & {TADEOT Jan 2020} & \hline
{ - Hardware integration} & {TADEOT Jan 2020} & \hline
{Dahoam-Flows} & {} & \hline
{ - Handling of registration aborted} & {TADEOT Jan 2020} & \hline
\end{tabular}


\pagebreak
\section{List of Abbreviations}

\begin{tabular}{|l|l|}
\hline
\cellcolor[gray]{0.5}\textcolor{white}{Abbreviation} & \cellcolor[gray]{0.5}\textcolor{white}{Explanation} &
{Dahoam} & {Name of the Project} & \hline
\end{tabular}

\pagebreak

\section{List of Literature}

\href{https://gitlab.htl-leonding.ac.at/MirrorHome/Core/wikis/home}{Dahoam-Wiki}

\end{document}  